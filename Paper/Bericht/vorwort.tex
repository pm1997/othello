\chapter*{Hinweise zur Arbeit}
\addcontentsline{toc}{chapter}{Hinweise zur Arbeit}
\section*{Verfügbarkeit der Implementierung}
Die Arbeit, ihre LaTex Quelldateien und die innerhalb der Arbeit diskutierte Implementierung von Othello sowie der besprochenen Agenten ist verfügbar unter: 
\\\href{https://github.com/pm1997/Othello}{https://github.com/pm1997/Othello}.
\\Die Arbeit selbst findet sich im Unterverzeichnis:
\\\href{https://github.com/pm1997/othello/tree/master/Paper/Bericht}{https://github.com/pm1997/othello/tree/master/Paper/Bericht}.
\\Die Implemntierung im Verzeichnis:
\\\href{https://github.com/pm1997/othello/tree/master/python}{https://github.com/pm1997/othello/tree/master/python}.
\\Um Zugriff auf das Repository zur erhalten wird ein Account bei \mxZitat{GitHub} und eine entsprechende Freischaltung durch Patrick Müller, den Eigentümer des Repositorys, benötigt.

\section*{Kennzeichnung der Autorenschaft}
Die Autorenschaft der einzelnen Abschnitte der Arbeit ist durch die Nennung des Autors zu Beginn des Abschnittes gekennzeichnet. Dabei kommen die folgenden Kennzeichnungen zum Einsatz:
\begin{itemize}
\item \authorpatrick\ Zur Kennzeichnung eines von Patrick Müller verfassten Abschnitts
\item \authormax\ Zur Kennzeichnung eines von Max Zepnik verfassten Abschnitts
\end{itemize}
Die Abschnitte werden nur gesondert gekennzeichnet wenn sich der Autor gegenüber der letzten Kennzeichnung geändert hat.  