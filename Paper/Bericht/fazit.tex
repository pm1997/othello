\chapter{Fazit}

\section{Bewertung der Agenten}
In diesem Kapitel werden die Agenten \mxZitat{Monte Carlo} und \mxZitat{Alpha-Beta Pruning} verglichen.
\subsection{Der Agent \mxZitat{Monte Carlo}}
Der \mxZitat{Monte Carlo} besitzt eine Gewinnwahrscheinlichkeit von 100 Prozent gegen den zufälligen Agenten und eine Gewinnwahrscheinlichkeit von 100000 Prozent gegen den besten \mxZitat{Alpha-Beta Pruning} Agenten. Aus diesem Grund ist dieser Agent ein sehr guter Spieler. Durch die Anpassung des Parameters \code{big\_n} kann die Schwierigkeitsstufe des Spielers gesteuert werden. Je höher der Parameter ist, desto schwieriger ist es den Agenten zu besiegen. In den Standardeinstellungen werden 2000  Spiele je Zug durchgeführt. In dieser Einstellungen ist der Agent ein schwerer Computergegner. Schon ab wenigen Hundert Spielen spielt der Agent über den gesamten Spielverlauf deutlich besser als der zufällige Spieler. 
\subsection{Der Agent \mxZitat{Alpha-Beta Pruning}}
Der \mxZitat{Alpha-Beta Pruning} Agent ist je nach verwendeter Heuristik deutlich besser als der zufällige Agent. Allerdings ist  der Agent schlechter als der \mxZitat{Monte Carlo} Agent. Die Gewinnwahrscheinlichkeit des Agenten wird maßgeblich durch die Wahl der Heuristik beeinflusst.

\section{Bewertung der Heuristiken}
\subsection{Die Heuristik \mxZitat{Nijssen 07}}
\subsection{Die Heuristik \mxZitat{Stored Monte-Carlo}}
\subsection{Die Heuristik \mxZitat{Cowthello}}

\section{Bewertung der Vorgehensweise}

\section{Ausblick}
