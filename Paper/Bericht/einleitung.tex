\chapter{Einleitung}
\authormax
%Umfeld - Othellos Gechichte
\mxZitat{\ot} ist eine Variante des, laut der Wikipedia, aus dem 19. Jahrhundert stammenden Spiels \mxZitat{Reversi} \cite{Wiki:EN:Reversi}. Als klassisches Denkspiel wird es von zwei Spielern gegeneinander gespielt. Die Beschreibung \mxZitat{Othello: A Minute to Learn, A Lifetime to Master} \cite{Rose} spiegelt dabei zwei wesentliche Aspekte, die es auszeichnen, wider:
\\Die Anzahl der einfach gehaltenen Regeln hält sich in Grenzen und dennoch stellt es eine gewisse Herausforderung dar. Damit wird es zu einem einfach erlernbaren und kurzweiligen Spielerlebnis, dass es dem Spieler gestattet sich mit mehr und mehr Kenntnissen und Übungen stetig zu verbessern.
\\Damit bietet sich das Spiel zur Umsetzung zu einem Spiel gegen den Computer an.
%Ziel der Arbeit
\paragraph{Das Ziel dieser Arbeit} besteht darin eine Anwendung zu erstellen, die es einem menschlichen Spieler gestattet gegen Computergegner \mxZitat{\ot} zu spielen. Eine Anforderung an den Computeragenten besteht darin, dass er dem menschlichen Gegenspieler eine gewisse Schwierigkeit gegenüberstellt.
\\Als Grundvoraussetzung für einen dieser Anforderung entsprechenden Computeragenten wird dabei die Tatsache betrachtet, dass der Agent zumindest besser spielt als jener Agent, der zufällig Züge durchführt.
%Betrachtungsgrenzen
\paragraph{Betrachtungsgrenzen}
Zur Beurteilung der Schwierigkeit, die ein Computergegner für einen menschlichen Spieler darstellt, bietet sich eine Gegenüberstellung der gewonnen Spiele im gemeinsamen Spiel an. Leider steht im Rahmen der Arbeit jedoch kein \ot\ Spieler zur Verfügung, der den Computeragenten auf einem gleichbleibend hohen Niveau testen könnte. Entsprechend fokussiert sich diese Arbeit auf den Vergleich des Computer Agenten mit einem zufällig spielenden Agenten bzw. den Vergleich verschiedener Algorithmen bzw. Spielstrategien untereinander.
%Aufbau der Arbeit
\newpage
\paragraph{Aufbau der Arbeit}
Die Arbeit ist dabei wie folgt aufgebaut:
\begin{enumerate}
\item Im Kapitel \ref{othello-chapter} werden die grundlegenden Begrifflichkeiten des Spiels sowie die Spielregeln eingeführt.
\item Danach werden im Kapitel \ref{basics} die zwei im weiteren Verlauf der Arbeit verwendeten Algorithmen mit ihren jeweiligen Variationen hergeleitet und erläutert.
\item Im Anschluss daran wird im Kapitel \ref{implementation} die im Rahmen der Arbeit entwickelte Implementierung der Spielmechanik sowie der Agenten vorgestellt und diskutiert.
\item Anhand der Implementierung werden die einzelnen Strategien dann evaluiert und bewertet bevor
\item im Kapitel \ref{Fazit} die Ergebnisse bewertet werden und ein Ausblick gegeben wird.
\end{enumerate}
