\chapter{Othello}
\label{othello-chapter}
\ot\ wird auf einem 8x8 Spielbrett von zwei Spielern gespielt. Es gibt je 64 Spielsteine, welche auf einer Seite schwarz, auf der anderen weiß sind. Der Startzustand besteht aus einem leeren Spielbrett, in welchem sich in der Mitte ein 2x2 Quadrat aus abwechselnd weißen und schwarzen Steinen befindet. Anschließend beginnt der Spieler mit den schwarzen Steinen.
\\Die Spielfelder werden in verschiedene Kategorien eingeteilt (siehe Abbildung
 \ref{fig:kat1}):
\begin{itemize}
\item Randfelder: äußere Felder (blaue Felder) \cite{wikibooks}
\item C-Felder: Felder, welche ein Feld horizontal oder vertikal von den Ecken entfernt sind (vgl. \cite{Berg})
\item X-Felder: Felder, welche ein Feld diagonal von den Ecken entfernt sind \cite{wikibooks}
\item Zentrum: innerste Felder von C3 bis F6 (grüne Felder) \cite{wikibooks}
\item Zentralfelder: Felder D4 bis E5 \cite{wikibooks}
\item Frontsteine: die äußersten Steine auf dem Spielbrett, welche andere Steine umschließen (vgl. \cite{Ortiz.}).
\end{itemize}
Diese Kategorien sind für die spätere Strategie wichtig.
\\\mxPicture{5.8cm}{kategorie}{Kategorien des Spielfeldes}{Kategorien des Spielfeldes}{fig:kat1}{}
\\Von \ot\ gibt verschiedene Varianten. Eine Variante ist Reversi.
Die verschiedenen Varianten sind allerdings bis auf die Startposition gleich. Bei der Variante Reversi sind die Zentralfelder noch nicht besetzt und die Spieler setzen die vier Steine selbst, während bei \ot\ die Startaufstellung fest vorgegeben ist.

\section{Spielregeln}

\ot\ besitzt wenige Spielregeln, welche im Spielverlauf aber auch taktisches oder strategisches Geschick erfordern. Jeder Spieler legt abwechselnd einen Stein auf das Spielbrett. Dabei sind folgende Spielregeln zu beachten, welche auch in Abbildung \ref{fig:turn2} abgebildet sind:
\begin{itemize}
\item Ein Stein darf nur in ein leeres Feld gelegt werden.
\item Es dürfen nur Steine auf Felder gelegt werden, welche einen oder mehrere gegnerischen Steine mit einem bestehenden Stein umschließen würden. Dies ist im linken Spielbrett durch grüne Felder und im mittleren Spielbrett durch das gelbe Feld hervorgehoben. Das gelbe Feld (F5) umschließt mit dem Feld D5 (blau) einen gegnerischen Stein. Es können auch mehrere Steine umschlossen werden. Allerdings dürfen sich dazwischen keine leeren Felder befinden.
\item Von dem neu gesetzten Stein in alle Richtungen ausgehend werden die umschlossenen generischen Steine umgedreht, sodass alle Steine die eigene Farbe besitzen. In dem Beispiel ist das im dem rechten Spielbrett zu sehen. F5 umschließt dabei das Feld E5 (rot). Dieses Feld wird nun gedreht und wird schwarz.
\item Ist für einen Spieler kein Zug möglich muss dieser aussetzen. Ein Spieler darf allerdings nicht freiwillig aussetzen, wenn noch mindestens eine Zugmöglichkeit besteht.
\item Der Spieler mit den meisten Steinen seiner Farbe gewinnt das Spiel.
\item Ist für beide Spieler kein Zug mehr möglich, ist das Spiel beendet. 
\item Das Spiel endet ebenfalls, wenn alle Felder des Spielbrettes besetzt sind.
\end{itemize}
\mxPicture{15cm}{turn2}{valide Zugmöglichkeiten für Schwarz und ausgeführter Zug}{valide Zugmöglichkeiten für Schwarz und ausgeführter Zug}{fig:turn2}{}
Aus diesen Regeln ergibt sich auch eine maximale Zuganzahl von 60 Zügen, da nach diesen alle leere Felder belegt sind.
\newpage
\section{Spielverlauf}
Das Spiel wird in drei Abschnitte eingeteilt \cite{Ortiz.}:
\begin{itemize}
\item Eröffnungsphase
\item Mittelspiel
\item Endspiel
\end{itemize}
Diese Abschnitte sind jeweils 20 Spielzüge lang.
Im Eröffnungs- und Endspiel stehen im Vergleich zum Mittelspiel wenige Zugmöglichkeiten zur Verfügung, da entweder nur wenige Steine auf dem Spielbrett existieren oder das Spielbrett fast gefüllt ist und nur noch einzelne Lücken übrig sind. Im Mittelspiel existieren viele Möglichkeiten, da sich schon mindestens 20 Steine auf dem Spielbrett befinden und diese sehr gute Anlegemöglichkeiten bieten. 
\section{Spielstrategien}
\label{strat1}
Wie in anderen Spielen gibt es auch in \ot\ verschiedene Strategien. Dabei kann beispielsweise offensiv gespielt werden, indem versucht wird möglichst viele Steine in einem Zug zu drehen. Es gibt auch defensive \mxZitat{stille} Züge. Ein \mxZitat{stiller} Zug dreht keinen Frontstein um und dreht möglichst nur wenige innere Steine um (vgl. \cite{Ortiz.}).
\\ Generell ist eine häufig genutzte Strategie die eigene Mobilität zu erhöhen und die Mobilität des Gegners zu verringern. Mit dem Begriff Mobilität sind die möglichen Zugmöglichkeiten gemeint. Durch das Einschränken der gegnerischen Mobilität hat dieser weniger Zugmöglichkeiten und muss so ggf. strategisch schlechtere Züge durchführen.
\\Die Position der Steine auf dem Spielbrett sollte ebenfalls nicht vernachlässigt werden.
Beispielsweise sollen Züge auf X-Felder vermieden werden, da der Gegner dadurch Zugang zu den Ecken bekommt. Dadurch können ggf. die beiden Ränder und die Diagonale gedreht werden und in den Besitz des Gegners gelangen.
\\In der Eröffnungsphase sollten die Randfelder ebenfalls vermieden werden, da diese in dieser frühen Phase des Spiels noch gedreht werden können und der taktische Vorteil in einen strategischen Nachteil umgewandelt wird.
\section{Eröffnungszüge}
\label{othello-eroff}
In der nachfolgenden Tabelle \ref{tab:eroeffnungen} sind verschiedene Spieleröffnungen und deren Häufigkeit in Spielen aufgelistet. Spielzüge werden in \ot\ durch eine Angabe der Position, auf welche der Stein gesetzt wird, dargestellt. Ein vollständiges Spiel lässt sich deshalb in einer Reihe von maximal 60 Positionen darstellen.

\begin{table}[h]
  \centering
  \begin{tabular}{| c | c | c |}
    \hline
     Name & Häufigkeit & Spielzüge \\ \hline
     Tiger  &   47\%   & f5 d6 c3 d3 c4 \\ \hline
     Rose &   13\%   & f5 d6 c5 f4 e3 c6 d3 f6 e6 d7 \\ \hline
     Buffalo &  8\%    & f5 f6 e6 f4 c3 \\ \hline
     Heath  &  6\%    &  f5 f6 e6 f4 g5 \\ \hline
     Inoue &   5\%   &  f5 d6 c5 f4 e3 c6 e6\\ \hline
     Shaman &   3\%   & f5 d6 c5 f4 e3 c6 f3 \\ \hline
  \end{tabular}
  \caption{Liste von Othelloeröffnungen \cite{Ortiz.}}
  \label{tab:eroeffnungen}
\end{table}
\cite{Ortiz.} gibt folgende weitere Tipps für Eröffnungen:
\begin{itemize}
\item Versuche weniger Steinchen zu haben als dein Gegner.
\item Versuche das Zentrum zu besetzen.
\item Vermeide zu viele Frontsteine umzudrehen.
\item Versuche eigene Steine in einem Haufen zu sammeln, statt diese zu verstreuen.
\item Vermeide vor dem Mittelspiel auf die Kantenfelder zu setzen.
\end{itemize}
Viele dieser Tipps können auch im späteren Spielverlauf verwendet werden.
