\addcontentsline{toc}{chapter}{Abstract}
\chapter*{Abstract}
\addcontentsline{toc}{section}{Deutsch}
\authormax

\mxZitat{\ot} ist eine Variante des Spiels \mxZitat{Reversi}. Als klassisches Denkspiel wird es von zwei Spielern gegeneinander gespielt. Damit bietet sich das Spiel zur Umsetzung zu einem Spiel gegen den Computer an.
\\Ziel dieser Arbeit ist die Entwicklung von computergesteuerten Agenten die einem menschlichen Spieler als Gegner zur Verfügung stehen.
\\Dabei wurden mehrer Agenten auf Grundlage der Grundlage der Strategien des \abab\ (engl. \mxZitat{\abp}) sowei des \mc\ Algorithmus entwicklet. Zur Bewertung einzelner Spielzustände werden, sofern gemäß der Strategie erforderlich, Heuristiken herangezogen.
\\Als eine Heuristik wurde unter anderem die durchschnittliche Gewinnwahrscheinlichkeit als Ergebnis einer einzlenen Spielentscheidung herangezogen. Bei der Umsetzung dieses Ansatzes werden Symmetrieeigenschaften des Spielfeldes ausgenutzt.
\\Die beschriebenen Agenten werden in der Programmiersprache \mxZitat{Python} implemntiert.
\\Zur Evaluierung werden die Ergebnisse der einzlenen Agenten im Spiel gegen einen zufällig agierenden Agenten, sowie untereinander betrachtet. Unter Verwendung der gesetzten Parameter kann dabei nur der auf dem \mc\ Algorithmus basierende Agent überzeugen.
\\Die auf der Betrachtung der durchschnittliche Gewinnwahrscheinlichkeit als Ergebnis einer einzlenen Spielentscheidung basierende Heuristik kann im Vergleich mit anderen Heuristiken nicht überzeugen.

\begin{center}
***
\end{center}