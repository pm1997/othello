\documentclass{report}
\input{fleqn.clo}
\usepackage[ngerman]{babel}
\usepackage{graphicx}
\usepackage{epstopdf}
\usepackage{epsfig}
\usepackage{a4wide}
\usepackage{amssymb}
\usepackage{fancyvrb}
\usepackage{alltt}
\usepackage{theorem}
%\usepackage[fleqn]{amsmath}
\usepackage{stmaryrd}
\usepackage{enumerate}
\usepackage{color}
\usepackage{xfrac}
\usepackage{amsmath}
\usepackage{qtree}

\usepackage{hyperref}
\usepackage[all]{hypcap}
\hypersetup{
	colorlinks = true, % comment this to make xdvi work
	linkcolor  = blue,
	citecolor  = red,
        filecolor  = Gold,
        urlcolor   = [rgb]{0.5, 0.0, 0.5},
	pdfborder  = {0 0 0}
}
\setlength{\mathindent}{1.3cm}
\setlength{\textwidth}{17cm}
\addtolength{\oddsidemargin}{-1cm}
\addtolength{\evensidemargin}{-1cm}
\addtolength{\topmargin}{-1cm}

\usepackage{fancyhdr}
\usepackage{lastpage}
\usepackage[utf8]{inputenc}
% \renewcommand*{\familydefault}{\sfdefault}

\pagestyle{fancy}

\fancyfoot[C]{--- \thepage/\pageref{LastPage}\ ---}

\fancypagestyle{plain}{%
\fancyhf{}
\fancyfoot[C]{--- \thepage/\pageref{LastPage}\ ---}
\renewcommand{\headrulewidth}{0pt}
}

\renewcommand{\chaptermark}[1]{\markboth{\chaptername \ \thechapter.\ #1}{}}
\renewcommand{\sectionmark}[1]{\markright{\thesection. \ #1}{}}
\fancyhead[R]{\leftmark}
\fancyhead[L]{\rightmark}

\definecolor{amethyst}{rgb}{0.2, 0.4, 0.6}
\definecolor{orange}{rgb}{1, 0.9, 0.0}
\definecolor{grey}{rgb}{0.5, 0.5, 0.5}

\newfont{\chess}{chess20}
\newfont{\bigchess}{chess30}
\newcommand{\chf}{\baselineskip20pt\lineskip0pt\chess}

\newcommand{\blue}[1]{{\color{blue}#1}}
\newcommand{\green}[1]{{\color{green}#1}}
\newcommand{\red}[1]{{\color{red}#1}}
\newcommand{\grey}[1]{{\color{grey}#1}}
\newcommand{\el}{\!\in\!}

% set the monospace-font to Inconsalata-g
% font-source: http://leonardo-m.livejournal.com/77079.html
%\renewcommand{\encodingdefault}{T1}
%\renewcommand{\ttdefault}{inconsolatag}

\newcommand{\ds}{\displaystyle}
\newcommand{\AI}{\textsc{Ai}}
\newcommand{\cnt}{\texttt{\#}}
\def\pair(#1,#2){\langle #1, #2 \rangle}
\newcommand{\myFig}[1]{Figure \ref{fig:#1} on page \pageref{fig:#1}}

{\theorembodyfont{\sf}
\newtheorem{Definition}{Definition}
\newtheorem{Theorem}[Definition]{Theorem}
\newtheorem{Proposition}[Definition]{Proposition}
\newtheorem{Lemma}[Definition]{Lemma}
\newtheorem{Corollary}[Definition]{Corollary}
}


%------------------------------------------------------------------------------------
% Farben definieren
%------------------------------------------------------------------------------------
\definecolor{hellgelb}{rgb}{1,1,1} 
\definecolor{colKeys}{rgb}{0,0,1} 
\definecolor{colIdentifier}{rgb}{1,0,0} 
%\definecolor{colComments}{rgb}{0.66,0.66,0.66} 
\definecolor{colComments}{rgb}{0.5,0.5,0.5} 
\definecolor{colString}{rgb}{0,0.5,0} 
\usepackage[onehalfspacing]{setspace}
\definecolor{mxBlue}{RGB}{0,33,99}
%-------------------------------------------------------------------------------------

\usepackage{listings}

\lstset{
    float=hbp,
    basicstyle=\ttfamily\footnotesize,
    identifierstyle=\color{colIdentifier},
    keywordstyle=\color{mxBlue},
    stringstyle=\color{colString},
    commentstyle=\color{colComments},
    columns=flexible,
    tabsize=3,
    frame=single,
    extendedchars=false,
    showspaces=false,
    showstringspaces=false,
    numbers=right,
    numberstyle=\tiny,
    breaklines=false,
    backgroundcolor=\color{hellgelb},
    breakautoindent=false,
    captionpos=b
} 
%-----------------------------------------------------------------------------------
% C++ definieren
%-----------------------------------------------------------------------------------
\lstdefinelanguage{CPP}{
  keywords={new, true, false, return, null, catch, switch, for, if, in, while, do, else, case, break, cout, cerr, void, int , Json,length,Value, string, or , == },
  keywordstyle=\color{mxBlue}\bfseries,
  ndkeywords={class, boolean,ui,user_config_j ,system_config_j , this},
  ndkeywordstyle=\color{gray}\bfseries,
  identifierstyle=\color{black},
  sensitive=false,
  comment=[l]{//},
  morecomment=[s]{/*}{*/},
  commentstyle=\color{colComments}\ttfamily,
  stringstyle=\color{red}\ttfamily,
  morestring=[b]',
  morestring=[b]"
}
%------------------------------------------------------------------------------------------------------------

% todo definieren ----------------------------------------------------
\usepackage[colorinlistoftodos,prependcaption,textsize=tiny]{todonotes}
\newcommand{\unsure}[1]{\todo[linecolor=red,backgroundcolor=red!25,bordercolor=red]{#1}}
\newcommand{\change}[1]{\todo[linecolor=blue,backgroundcolor=blue!25,bordercolor=blue]{#1}}
\newcommand{\info}[1]{\todo[linecolor=green,backgroundcolor=green!25,bordercolor=green]{#1}}
\newcommand{\improvement}[1]{\todo[linecolor=lime,backgroundcolor=lime!25,bordercolor=lime]{#1}}
\newcommand{\thiswillnotshow}[1]{\todo[disable]{#1}}

%----Wörter formatieren-----------------------------------
\newcommand{\ot}{\textit{Othello}}
\newcommand{\gtheorie}{Spieltheorie}
\newcommand{\board}{Board}
\newcommand{\state}{Spielzustand}

\newcommand{\code}[1]{\mxZitat{\texttt{#1}}} 
\newcommand{\mxZitat}[1]{\glqq{}#1\grqq{}} 

\vbadness=\maxdimen

\newcommand{\mxPicture}[5]{\XPicture{#1}{#2}{#3}{#4}{#5}}

\newcommand{\XPicture}[5]{\mxMinipage{\centering \vspace*{10pt} \XGraphic{#1}{#2} \captionof{figure}[#3]{#4} \vspace*{10pt} \label{#5} \vspace*{20pt} }}
\newcommand{\mxMinipage}[1]{\begin{minipage}{\textwidth} #1 \end{minipage}}
\newcommand{\XGraphic}[2]{\includegraphics[width=#1]{#2}}

\graphicspath{{./Bilder/}}

\newcounter{exercise}

\title{Othello \\[0.3cm]
      --- Entwicklung einer KI für das Spiel ---}
\author{Patrick Müller, Max Zepnik}
\date{\today}

\newcommand{\so}[0]{$S\textsubscript{0}$}
\newcommand{\abp}{Alpha-Beta Pruning}

% später entfernen!! ---------------------------------------
\setlength {\marginparwidth }{2cm}
% ----------------------------------------------------------

\begin{document}
\maketitle
\tableofcontents

\chapter{Einleitung}

Computergegner \info{am Ende schreiben} 
..\\.
test..


text1
\unsure{auf Fazit beziehen?}
...
\chapter{Grundlagen}
\section{Spieltheorie}
test23 \cite{Russell.2016}


\begin{Definition}[Game]
  A \blue{game} is a tuple of the form
  \\[0.2cm]
  \hspace*{1.3cm}
  $\mathcal{P} = \langle Q,\mathtt{nextStates}, \mathtt{start}, \mathtt{goal}\rangle$
...
\change{falsche Definition}

....

Heuristiken erklären 
\improvement{zuerst Definitionen aufschreiben}
...
\end{Definition}
\chapter{Othello}
\label{othello-chapter}
\info{Othello erklären}
Von Othello gibt verschiedene Varianten. Eine Variante ist Reversi.
\info{weiter}
\\Othello wird auf einem 8x8 Spielbrett mit zwei Spielern gespielt. Es gibt je 64 Spielsteine, welche auf einer Seite schwarz, auf der anderen weiß sind. Der Startzustand besteht aus einem leeren Spielbrett, in welchem sich in der Mitte ein 2x2 Quadrat aus abwechselnd weißen und schwarzen Steinen befindet. Anschließend beginnt der Spieler mit den schwarzen Steinen.
\section{Spielregeln}
Othello besitzt einfache Spielregeln, welche im Spielverlauf aber auch taktisches oder strategisches Geschick erfordern. Jeder Spieler legt abwechselnd einen Stein auf das Spielbrett. Dabei sind folgende Spielregeln zu beachten:
\begin{itemize}
\item Ein Stein darf nur in ein leeres Feld gelegt werden.
\item Auf mindestens einer Seite des Steines (vertikal, horizontal oder diagonal) muss ein gegnerischer Stein durch diesen Stein umschlossen werden. Dies bedeutet, dass nach dem angrenzenden gegnerischen Stein beliebig viele generische Steine folgen und durch einen eigenen Stein abgeschlossen werden. Es dürfen sich keine leere Felder dazwischen befinden.
\item Die umschlossenen generischen Steine werden umgedreht, sodass alle Steine die eigene Farbe besitzen.
\item Ist für beide Spieler kein Zug mehr möglich, ist das Spiel beendet. Der Spieler mit den meisten Steinen seiner Farbe gewinnt das Spiel.
\end{itemize}
\section{Spielverlauf}
Das Spiel wird in drei Abschnitte eingeteilt:
\begin{itemize}
\item Eröffnungsphase
\item Mittelspiel
\item Endspiel
\end{itemize}
Diese Abschnitte sind jeweils 20 Spielzüge lang.
Im Eröffnungs- und Endspiel stehen zum Mittelspiel wenige Zugmöglichkeiten zur Verfügung, da entweder nur wenige Steine auf dem Spielbrett existieren oder das Spielbrett fast gefüllt ist und nur noch einzelne Lücken übrig sind. Im Mittelspiel existieren sehr viele Möglichkeiten, da sich schon mindestens 20 Steine auf dem Spielbrett befinden und diese sehr gute Anlegemöglichkeiten bieten. 
\section{Spielstrategien}
\section{Eröffnungszüge}
\info{Algo erklären}
\chapter{Implementierung der KI}
Die theoretischen Grundlagen werden nun angewandt und in ein Python \ot\ Spiel implementiert. \unsure{anders?}
\section{Vorgehensweise}
In den folgenden Unterkapiteln  werden verschiedene Spielalgorithmen vorgestellt und implementiert. Anschließend werden diese verbessert und auch unter Berücksichtigung des Laufzeitverhaltens analysiert.
\\Zunächst wird aber die grundsätzliche Programmstruktur erläutert und das Spielgerüst implementiert, damit unterschiedliche Spieler es ausführen können.
\section{Vorbereitungen}
Die Python Implementierung befindet sich in dem Ordner \mxZitat{python}. Die verschiedenen Komponenten des Spiels wurden nach Komponenten gruppiert auf mehrere Dateien aufgeteilt.
\\Das Spiel wird durch \mxZitat{main-game.py} gestartet.
Diese Datei führt mehrere Benutzerabfragen nach den Spielern aus, ermittelt den zu spielenden Zug und gibt dann das Spielbrett in der Konsole aus.
\\Die Hauptklasse ist \mxZitat{Othello}. Diese speichert u.a das Spielbrett, den aktiven Spieler und die durchgeführten Züge. In dieser Klasse befinden sich alle \ot-spezifischen Funktionen, welche von keinem Spieler abhängen. Dazu zählt beispielsweise das Ermitteln der möglichen nächsten Spielzüge und das Spielen eines vorgegebenen Zuges.
\\Aus den möglichen Zügen wählen die verschiedenen Spieler den für sie jeweils besten Zug aus. Die Spieler befinden sich in dem Unterordner \mxZitat{Players}.
\section{Spieler}
Es gibt folgende \mxZitat{Haupt}-Spieler:
\begin{enumerate}
\setcounter{enumi}{-1}
\item Human Player
\item Random Player
\item Monte Carlo
\item Alpha-Beta Pruning
\item Machine Learning
\end{enumerate}
Spieler 0 wird für die manuelle Eingabe von Zügen eingesetzt. Alle anderen Spieler sind Computerspieler und spielen automatisch.
Nachfolgend werden die einzelnen Computerspieler kurz beschrieben.

\subsection{Random}
Der Spieler Random wählt aus der Liste der möglichen Züge zufällig einen Zug aus und gibt diesen an die Hauptfunktion zurück.

\subsection{Monte Carlo}
Dieser Spieler verwendet den in Kapitel \ref{mc_algo} verwendeten Algorithmus um den Zug mit der höchsten Gewinnwahrscheinlichkeit auszuwählen. Dazu spielt der Spieler zufällig eine bei Spielstart eingestellte Anzahl an Spielen ab der aktuellen Spielsituation und berechnet daraus den Anteil der gewonnen Spiele je verfügbaren Zug. Den Zug mit der höchsten Gewinnwahrscheinlichkeit wird nun im \mxZitat{realer} Zug des Spielers ausgewählt.

\subsection{Alpha-Beta Pruning}
Ebenso wie der Spieler \mxZitat{Monte Carlo} wird der Spielalgorithmus im Theorieteil erläutert. Der beste Zug wird dadurch berechnet, dass eine eingeschränkte Breitensuche bis zu einer bestimmten Tiefe durchgeführt wird, dabei allerdings auch der Gegenspieler beachtet wird.
Statt einer kompletten Tiefensuche mit MiniMax werden Züge mit einer geringen Zugwahrscheinlichkeit nicht evaluiert. Die Grundidee des Algorithmus ist, dass sowohl der aktuelle Spieler, als auch der Gegenspieler jeweils den für sie besten Zug und für den Gegner schlechtesten Zug spielen.
\\Nach der eingeschränkten Breitensuche können mehrere Möglichkeiten gewählt werden.
Es existieren einerseits mehrere Heuristiken, andererseits können auch andere Spieler ab diesen Spielzügen das Spiel berechnen. Diese Möglichkeiten werden in dem Kapitel \ref{ab_comb} genauer erläutert.

\subsection{Machine Learning}
Dieser Spielalgorithmus besteht hauptsächlich aus dem Monte Carlo Spieler. Der deutliche Unterschied zu diesem besteht aber in der Auswahl der zu simulierenden Spiele. Während Monte Carlo zufällig einen Zug aus den verfügbaren Zügen auswählt, verwendet Machine Learning eine gewichtete Zufallsfunktion. Der Algorithmus speichert die gespielten und gewonnen Spiele in einer Datenbank. Bei der Auswahl des Zuges, gewichtet er die Zugmöglichkeiten, die eine höhere Gewinnwahrscheinlichkeit besitzen höher als die Züge mit einer geringeren Wahrscheinlichkeit. Dadurch wählt er statistisch die Züge mit einer höheren Gewinnwahrscheinlichkeit aus. Durch das Speichern der Ergebnisse der simulierten Spiele in der Datenbank, \mxZitat{lernt} die Datenbank mit der Zeit dazu und kann in weiteren Spielen zuverlässigere Wahrscheinlichkeiten zurückgeben.

\section{Kombination von Monte-Carlo und Alpha-Beta Abschneiden}
\label{ab_comb}
\chapter{Evaluierung}
Ziel dieses Kapitel ist es die im vorherigen Kapitel vorgestellte Implementierung zu testen. Dabei ergeben sich im wesentlichen zwei Testaspekte: Zum einen die vorgestellten Agenten und zum anderen die Bedeutung der in Kapitel \ref{impl:stored-mc} eingeführten Feldkategorien.
\paragraph{}
Die Evaluierung der einzelnen Agenten erfolgt dabei im wesentlichen dadurch die Agenten gegeneinander spielen zu lassen. Mittels einer genügend großen Anzahl von Spielen lässt sich darart ermitteln welcher der jeweils verwendeten Agenten im Mittel die meisten Spiele gewinnt und damit besser ist.
\\Die Bedeutung der Feldkategorien wird Anhand der Gewinnwahrscheinlichkeit einer Spielkategorie über den Spielfortschritt untersucht.
\section{Evaluierung der einzelnen Agenten}
\label{cpt:eval-agents}
Nun gilt es die in Kapitel \ref{implementation} beschriebenen Agenten zu evaluieren. In Ermangelung eines Othello-Spielers, der über ein entsprechend gute Spielfähigkeiten verfügt, so dass er zu einem aussagekräftigen Vergleich herangezogen werden könnte, werden die einzelnen Agenten untereinander verglichen. Die einstellbaren Parameter wurden dabei so gewählt, dass der Agent im Laufe eines Spiels ungefähr 5 Minuten zum Treffen seiner Entscheidung benötigt. Auf die genauen Werte der Parameter soll hier nicht näher eingegangen werden. Aus Gründen der Nachvollziehbarkeit finden sie sich in den Anschnitten \nameref{eval:agents:params:subsec-mc} bzw. \nameref{eval:agents:params:subsec-ab}
\\Bei der Bestimmung der Parameter sich gezeigt, dass die beim Agenten \mxZitat{Alpha-Beta Pruning} gewählte Heuristik die Gewinnchance des Agenten wesentlich beeinflusst. Aus diesem Grund wird der \mxZitat{Alpha-Beta Pruning}-Agent mit allen zur Verfügung stehenden Heuristiken seperat betrachtet. Die sich daraus Ergebenden Vergleiche sind in Tabelle \ref{tbl:cmp-agents} dargestellt. Der Agent \mxZitat{Alpha-Beta Pruning} wird dabei durch \mxZitat{AB} abgekürzt.
\paragraph{Vergleich mit dem Agent \mxZitat{Random}}
Grundsätzlich gilt, dass der durch einen Agenten zur Auswahl von Spielzügen benötigte Rechenaufwand nur dann gerechtfertigt ist, wenn sich dieser in irgendeinen Vorteil bietet, also besser spielt als ein Agent der beliebige Spielzüge durchführt. Aus diesem Grund werden alle Agenten zunächst mit dem \mxZitat{Random}-Agenten verglichen. Daraus ergeben sich zunächst die Vergleiche 1 bis 5.
\\Da ein Agent möglicherweise über bessere Gewinnchancen verfügt, wenn er als ein bestimmter Spieler auftritt gilt es außerdem zu Prüfen wie sich die Agenten verhalten, wenn sie auf der anderen Position verwendet werden. Daraus ergeben sich die Vergleiche 6 bis 10. 

\paragraph{Vergleich der Agenten \mxZitat{Alpha-Beta Pruning} und \mxZitat{Monte Carlo}}
Bisher wurden die Agenten \mxZitat{Alpha-Beta Pruning} und \mxZitat{Monte Carlo} lediglich mit dem \mxZitat{Random}-Agenten verglichen. Interessant ist daher auch die Fragestellung wie die beiden Agenten gegeneinander Spielen. Hier sei vorweggegriffen, dass Anhand der im Abschnitt \nameref{p:vgl-result} dargestellten Ergebnisse der Vergleiche 2 bis 5 bzw. 7 bis 10 festgestellt werden konnte, dass der Agent \mxZitat{Alpha-Beta Pruning} mit der Heuristik \mxZitat{???} \unsure{ermitteln} die besten Ergebnisse erzielt. Daher wird diese für den Vergleich der Agenten untereinander herangezogen. Aus den oben beschriebenen Gründen ergeben sich daraus die beiden Vergleiche 11 und 12.

\begin{table}[ht]
\begin{center}
\begin{tabular}{| c | c | c |} \hline
Vergleich & Agent 1 & Agent 2 \\ \hline
\hline
0 & Random & Random  \\ \hline
\hline
 1 & Monte Carlo                         & Random  \\ \hline
 2 & AB (Nijssen 2008 Heuristik)         & Random\\ \hline
 3 & AB (Stored Monte Carlo Heuristik)   & Random\\ \hline
 4 & AB (Cowthello Heuristik)            & Random\\ \hline
 5 & AB (mit anschließendem Monte Carlo) & Random\\ \hline
\hline
 6 & Random                              & Monte Carlo                         \\ \hline
 7 & Random                              & AB (Nijssen 2008 Heuristik)         \\ \hline
 8 & Random                              & AB (Stored Monte Carlo Heuristik)   \\ \hline
 9 & Random                              & AB (Cowthello Heuristik)            \\ \hline
10 & Random                              & AB (mit anschließendem Monte Carlo) \\ \hline
\hline
11 & AB (beste)                          & Monte Carlo                         \\ \hline
12 & Monte Carlo                         & AB (beste)                          \\ \hline
\end{tabular}
\end{center}
\caption{Durchgeführte Vergleiche}
\label{tbl:cmp-agents}
\end{table}

\paragraph{Ergebnisse der Vergleiche}
\label{p:vgl-result}
Um zum einen ein möglichst aussagekräftiges Ergebnis zu erhalten und zum Anderen die zur Durchführung der Spiele erforderliche Rechenzeit zu beschränken wurden für jeden der in Tabelle \ref{tbl:cmp-agents} angegebenen Vergleiche jeweils bis zu 1000 Spiele durchgeführt. Bei einigen Kombinationen wurden aus Gründen der erforderlichen Rechenzeit weniger als 1000 Vergleiche durchgeführt. Ist dies der Fall, so wird dies in der entsprechenden Spalte angegeben. Die Ergebnisse dieser Vergleiche finden sich in Tabelle \ref{tbl:cmp-results}:

\begin{table}[ht]
\begin{center}
\begin{tabular}{| c | c | c | c | c | c |} \hline
 &  & \multicolumn{2}{|c|}{Gewonnene Spiele} & \multicolumn{2}{|c|}{$\varnothing$ Rechenzeit je Spiel [s]} \\ \hline
Vergleich & Anz. Spiele & Agent 1 & Agent 2 & Agent 1 & Agent 2 \\ \hline
\hline
 0 & 1000 & 429  &  527 &   0,00027 &   0,00026 \\ \hline
 \hline
 1 & 1000 & 1000 &    0 & 118,21435 &   0,00235 \\ \hline
 2 & -    & -    & -    & -         & -         \\ \hline
 3 & T.M. & -    & -    & -         & -         \\ \hline
 4 & 1000 &  564 &  324 & 156,59706 &   0,00056 \\ \hline
 5 & P.M. & -    & -    & -         & -         \\ \hline
 \hline
 6 & 1000 &    0 & 1000 &   0,00235 & 120,86694 \\ \hline
 7 & M.Z, & -    & -    & -         & -         \\ \hline
 8 & 1000 &  446 &  550 &   0,00034 & 234,01762 \\ \hline
 9 & 1000 &  289 &  709 &   0,00049 & 149,17649 \\ \hline
10 & S.K. & -    & -    & -         & -         \\ \hline
\hline
11 & -    & -    & -    & -         & -         \\ \hline
12 & -    & -    & -    & -         & -         \\ \hline
\end{tabular}
\end{center}
\caption{Ergebnisse der Vergleiche}
\label{tbl:cmp-results}
\end{table}

\subsection*{Parameter des \mxZitat{Monte Carlo}-Agenten}
\label{eval:agents:params:subsec-mc}

\subsection*{Parameter des \mxZitat{Alpha-Beta Pruning}-Agenten}
\label{eval:agents:params:subsec-ab}


\section{Evaluierung der Bedeutung verschiedener Feldkategorien}
Einleitung
\paragraph{Methode}
Für die Beurteilung der Bedeutung einzelner Feldkategorien wurde die für die im Kapitel \ref{cpt:eval-agents} durchgeführten Vergleiche mit der \mxZitat{Stored Monte Carlo Heuristik} aufgebaute Datenbank verwendet.
\\Diese Datenbank wurde wie in Kapitel \ref{heuristic} erläutert durch das Spielen von Partien zwischen zwei \mxZitat{Random}-Agenten aufgebaut. Im vorliegenden Fall wurden $140000$ Spiele simuliert. Nun kann für jede Zugnummer und für jeden Spieler die Gewinnwahrscheinlichkeit für das Spielen einer Feldkategorie berechnet werden in dem für jede Kategorie die Anzahl der Spiele die nach Spielen eines Feldes dieser Kategorie gewonnen wurde durch die Anzahl der Spiele die insgesamt für die jeweilige Kategorie gespielt wurde dividiert wird. Um eine Division durch $0$ zu vermeiden, wird die berechnete Gewinnwahrscheinlichkeit auf $0$ gesetzt sofern bei der jeweiligen Zugnummer nie eine derartige Feldkategorie gespielt wurde. Aufgrund der großen Anzahl von zufällig gespielten Spielen ist in diesem Fall davon auszugehen, dass es zu der jeweiligen Zugnummer keine Möglichkeit gibt eine derartige Feldkategorie zu spielen. 
\\Insgesamt ergibt sich damit pro Spieler eine Relation $\lbrace1, ..., 60\rbrace\times\lbrace0, ..., 8\rbrace\mapsto\lbrack0,1\rbrack$ die für die Zugnummer und für die und die Feldkategorie jeweils die Wahrscheinlichkeit angibt.
  
\paragraph{Ergebnisse}
Um die große Anzahl von $60\ \mathtt{Zugnummern} * 9\ \mathtt{Zugkategorien} = 540$ Wahrscheinlichkeitswerten pro Spieler übersichtlich darstellen zu können werden diese grafisch aufbereitet. Da hierfür ausschließlich zweidimensionale Diagramme zum Einsatz kommen sollen besteht die Möglichkeit hier nach der Zugnummer oder der Feldkategorie zu schneiden. Die Abbildungen \ref{fig:win-pro-fc-0} bis \ref{fig:win-pro-fc-8} im Anhang geben die Gewinnwahrscheinlihckeiten für den Schnitt nach der Feldkategorie an.
\\Die Abbildungen \ref{fig:win-pro-turn-0} bis \ref{fig:win-pro-turn-59} im Angang geben die Gewinnwahrscheinlichkeiten für den Schnitt nach der Zugnummer an. Da aus Platzgründen nicht alle 60 der sich daraus ergebenden Abbildungen abgedruckt werden sollen, wurde dabei die willkürlich gewählte Schrittweite von 5 Zugnummern gewählt. Die Situation für den Zug 59 wird trotzdem angegeben.
\paragraph{Beurteilung der Ergebnisse}
\chapter{Fazit}

\section{Bewertung der Agenten}
\subsection{Der Agent \mxZitat{Monte Carlo}}
\subsection{Der Agent \mxZitat{Alpha-Beta Pruning}}

\section{Bewertung der Heuristiken}
\subsection{Die Heuristik \mxZitat{Nijssen 07}}
\subsection{Die Heuristik \mxZitat{Stored Monte-Carlo}}
\subsection{Die Heuristik \mxZitat{Cowthello}}

\section{Bewertung der Vorgehensweise}

\section{Ausblick}


% später entfernen!! ----------------------------------------
\newpage
\listoftodos[Notes]
% -----------------------------------------------------------

\bibliographystyle{alpha}
\bibliography{quellen}

\end{document}
