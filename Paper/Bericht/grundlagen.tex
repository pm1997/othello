\chapter{Grundlagen}
\section{Spieltheorie}
\info{Einleitung fehlt}
\begin{Definition}[Spiel (Game)]
Ein \blue{Game} besteht aus einem Tupel der Form \\[0.2cm]
  \hspace*{1.3cm}
  $\mathcal{G} = \langle S\textsubscript{0},\mathtt{player}, \mathtt{actions}, \mathtt{result}, \mathtt{terminalTest}, \mathtt{utility} \rangle$
\\\so\ beschreibt den Startzustand des Spiels.
\\\textsc{player} ist auf der Menge der Spieler definiert und gibt den aktuellen Spieler zurück.
\\\textsc{actions} gibt die validen Folgezustände eines gegeben Zustands zurück.
\\\textsc{result} definiert das Resultat einer durchgeführten Aktion a und in einem Zustand s.
\\\textsc{terminalTest} prüft ob ein Zustand s ein Terminalzustand, also Endzustand, darstellt.
\\\textsc{utility} gibt einen Zahlenwert aus den Eingabenwerten s ( Terminalzustand) und p (Spieler) zurück. \\Positive Werte stellen einen Gewinn, negative Werte einen Verlust dar.
\info{quelle S.162}
\info{Definition States davor}
\end{Definition}
Eine spezielle Art von Spielen sind \blue{Nullsummenspiele}.
\begin{Definition}[Nullsummenspiele]
In einem \blue{Nullsummenspiel} ist die Summe der utility Funktion eines Zustands über alle Spieler 0. Dies bedeutet, dass wenn ein Spieler gewinnt mindestens ein Gegenspieler verliert.
\end{Definition}
Durch den Startzustand \so\ und der Funktion \textsc{action} wird ein \blue{Spielbaum (Game tree)} aufgespannt.
\begin{Definition}[Spielbaum (Game Tree)]
Ein \blue{Spielbaum} besteht aus einer einzigen Wurzel, welche einen bestimmten Zustand (meistens \so) darstellt. Die Kindknoten der Wurzel stellen die durch \textsc{actions} erzeugten Zustände dar. Die Kanten zwischen der Wurzel und den Kindknoten stellen jeweils die durchgeführte Aktion dar, die ausgeführt wurde um vom State s zum Kindknoten zu gelangen.
\end{Definition}
\begin{Definition}[Suchbaum (Search Tree)]
Ein \blue{Suchbaum} ist ein Teil des Spielbaums.
\end{Definition}
\cite{Russell.2016}
\info{Überleitung einfügen}
\section{Spielstrategien}
Es gibt verschiedene Spielstrategien. Im Folgenden werden diese kurz erläutert und anschließend verglichen.
\subsection{Min-Max}
\info{Patrick}
Der erste hier erläuterte Strategie ist der Min-Max Algorithmus. Dieser ist folgendermaßen definiert:
\\\Tree [.A [.B [.C eins ] [.D zwei ] ].B [.E {3 und 4} ] ].A
\subsection{Alpha-Beta Pruning}
\info{Max}
Der Min-Max Algorithmus berechnet nach dem Prinzip "depth-first" stets den kompletten Gametree. Bei der Betrachtung des Entscheidungsverhaltens des Algorithmus fällt jedoch schnell auf, dass ein nicht unerheblicher Teil aller möglichen Züge gar nicht erst in betracht gezogen wird. Dies geschieht aufgrund der Tatsache, dass diese Züge in einem schlechteren Ergebnis resultieren würden als die letzendlich ausgewählten.\newline
Dem \abp\ Algorithmus liegt der Gedanke zugrunde, dass die Zustände, die in einem realen Spiel nie auftreten würden auch nicht berechnet werden müssen. Damit steht die dafür regulär erforderliche Rechenzeit und der entsprechende Speicher dafür zur Verfügung andere, vielversprechendere Zweige zu verfolgen.
\subsubsection{Umsetzung}
Um die Umsetzung zu verdeutlcihen betrachten wir den folgenden, einfachen, Gametree. Unterhalb des Knotennamens wird jeweils das Ergebnis der MinMax-Funktion des Min-Max Algorithmus dargestellt.
\begin{figure}[h]
\caption[]{Beispielhafter Gametree}
\Tree [.{A\\a} [.{B\\5} [.{E\\5} ].{E\\5} [.{F\\13} ].{F\\13} [.{G\\8} ].{G\\8} ].{B\\5} [.{C\\6} [.{H\\h} ].{H\\h} [.{I\\i} ].{I\\i} [.{J\\j} ].{J\\j} ].{C\\c} [.{D\\d} [.{K\\k} ].{K\\k} [.{L\\l} ].{L\\l} [.{M\\m} ].{M\\m} ].{D\\d} ].{A\\a}
\end{figure}
\subsubsection{Ordnung der Züge}


\subsection{imperfect real time-time decisions}
\subsubsection{Evaluation functions}
\info{max}
\subsubsection{Cutting off search}
\info{max}
\subsubsection{Forward pruning}
\info{patrick}
\subsubsection{Search versus lookup}
\info{patrick}