\chapter{Grundlagen}
\label{basics}
In diesem Kapitel werden die theoretischen Grundlagen für eine Künstliche Intelligenz für das Spiel \ot\ erläutert.\\
Zunächst werden dazu einige Begriffe der Spieltheorie eingeführt. Diese werden dann zur Beschreibung einiger deterministischer Algorithmen zum Treffen einer Spielentscheidung verwendet. Den Abschluss des Kapitels bildet schließlich die Beschreibung einer stochastische Methode für diesen Zweck. 
\section{Spieltheorie}
In dem folgenden Unterkapitel werden grundlegende Definitionen eingeführt. Diese sind an der Arbeit \mxZitat{Playing Othello Using Monte Carlo} von J.A.M. Nijssen (\cite{Russell.2016}) angelehnt.
\begin{Definition}[Spiel (Game)(vgl. \cite{Russell.2016} S. 162)]
Ein \blue{Spiel} besteht aus einem Tupel der Form \\[0.2cm]
  \hspace*{1.3cm}
  $\mathtt{G} = \langle Q, S\textsubscript{0},\mathtt{player}, \mathtt{actions}, \mathtt{result}, \mathtt{terminalTest}, \mathtt{utility} \rangle$
\\Sei $Q$ die Menge aller Zustände im Spiel. Ein Spielzustand besteht aus dem aktuellen Spielbrett, der Zugnummer, dem aktiven Spieler und weiteren Parametern, welche zur genauen Darstellung eines Spielzustands führen.
\begin{itemize}
\item $\mathtt{Q}:$ Menge aller Spielzustände
\item $\so\ \in \mathtt{Q}$ beschreibt den Startzustand des Spiels.
\item $\mathtt{Players}:$ Menge aller Spieler: $\mathtt{Players} = \{S,W\}$
\item $\mathtt{player}: \mathtt{Q} \rightarrow \mathtt{Players}$ ist auf der Menge der Spielzustände definiert und gibt den aktuellen Spieler p zurück.
\item $\mathtt{actions}: \mathtt{Q} \rightarrow 2^\mathtt{moves}$ gibt die Menge der validen Zugmöglichkeiten  eines gegeben Zustands zurück. 
\\$\mathtt{moves}$ ist die Menge aller Zugmöglichkeiten.
\\$\mathtt{moves} = \{ <X, Y> \mid X \in \{0 .. 7\} \wedge Y \in \{ 0 .. 7 \} \} $
\\Diese geben die Koordinate der Zugposition des neuen Steines auf dem Spielbrett an. 
\item $\mathtt{result}:\mathtt{Q} \times \mathtt{actions} \rightarrow \mathtt{Q}$ definiert das Resultat einer durchgeführten Aktion a und in einem Zustand s.
\item $\mathtt{terminalTest}: \mathtt{Q} \rightarrow \mathbb{B}$ prüft ob ein Zustand s ein Terminalzustand \\($s \in \mathtt{terminalStates}$), also Endzustand, darstellt.
\\$ \mathtt{terminalTest}(s) = \begin{cases} \mathtt{True}\ \mid s \in \mathtt{terminalStates}$ 
$\\\mathtt{False}\ \mid s \not\in \mathtt{terminalStates} \end{cases}$
\item $\mathtt{utility}: \mathtt{terminalStates} \times \mathtt{Players} \rightarrow \{-1, 0, 1 \}$ gibt einen Zahlenwert aus den Eingabewerten s (Terminalzustand) und p (Spieler) zurück. \\Positive Werte stellen einen Gewinn, negative Werte einen Verlust dar. \mxZitat{0} stellt ein Unentschieden dar.
\end{itemize}
\end{Definition}
Eine spezielle Art von Spielen sind \blue{Nullsummenspiele}.
\begin{Definition}[Nullsummenspiele (vgl. \cite{Russell.2016} S. 161)]
In einem \\\blue{Nullsummenspiel} ist die Summe der utility Funktion eines Endzustandes ($\mathtt{terminalState}$) über alle Spieler 0. Es gilt also:
\vspace{0.2cm}
\\$\forall s \in \mathtt{terminalStates}: \sum\limits_{p \in Players} utility(s, p) = 0$
\vspace{0.2cm}
%\\Dies bedeutet, dass wenn ein Spieler gewinnt mindestens ein Gegenspieler verliert.
\\In \ot\ spielen zwei Spieler gegeneinander. Es gibt also nur die drei Möglichkeiten:
\begin{itemize}
\item Weiß gewinnt, Schwarz verliert
\item Schwarz gewinnt, Weiß verliert
\item Unentschieden
\end{itemize}
\end{Definition}
Durch den Startzustand $\so$ und der Funktion $\mathtt{actions}$ und $\mathtt{result}$ wird ein \blue{Spielbaum (Game Tree)} aufgespannt.
\begin{Definition}[Spielbaum (Game Tree)(vgl. \cite{Russell.2016} S. 162)]
Ein \blue{Spielbaum} \\besteht aus einer Wurzel, welche einen bestimmten Zustand (Startzustand $\so$) darstellt.\\ Die Kindknoten der Wurzel stellen die durch $\mathtt{actions}$ erzeugten Zustände dar. Die Kanten zwischen der Wurzel und den Kindknoten stellen jeweils die durchgeführte Aktion dar, die ausgeführt wurde um vom Zustand $\mathtt{s}$ zum Kindknoten $\mathtt{s'}$ zu gelangen. Diese Kindknoten können wiederum weitere Knoten enthalten oder ein Endpunkt des Baumes (Blatt = Spielende) %\improvement{umformulieren ?}
 darstellen.
\\Die mathematische Definition wird rekursiv durchgeführt:
\\$\mathtt{Spielbaum}: \mathtt{Q} \times \mathtt{List(actions)} \times \mathtt{List(Spielbaum)}$
\\$\mathtt{Node}(S,M,T) \in \mathtt{Spielbaum}$ g.d.w:
\begin{itemize}
\item $S \in Q \cup \{\Omega\}$
\item $M = [m\textsubscript{1}, .. , m \textsubscript{n}] \in \mathtt{List(actions)}$, wobei gilt:  $\forall i \in \{1 .. n\}: m\textsubscript{i} \in \mathtt{actions}(S) $
\item $T = [t\textsubscript{1}, .. , t \textsubscript{n}] \in \mathtt{List(Spielbaum)}$, wobei gilt: $ \forall i \in \{1 .. n\}: t\textsubscript{i} = \mathtt{result}(S, m\textsubscript{i}) $
\end{itemize}
\end{Definition}
\begin{Definition}[Suchbaum (Search Tree)(vgl. \cite{Russell.2016} S. 163)]
Ein \blue{Suchbaum} ist ein Teil des Spielbaums. Die Wurzel des Spielbaums besteht aus dem aktuellen Spielzustand. Alle Kindknoten und Blätter entsprechen dem Spielbaum beginnend ab dem aktuellen Zustand. Es gilt also:
\\ $B \in A \wedge B = A[s]$ mit $ A \in \mathtt{Spielbaum}, B \in \mathtt{Suchbaum}(s), s \in Q$
\end{Definition}
Mit den nun eingeführten Grundlagen der Spieltheorie werden im Folgenden einzelne Spielstrategien betrachtet. 
\newpage
\section{Spielstrategien}
Es gibt verschiedene Spielstrategien. Im Folgenden werden diese kurz erläutert und anschließend verglichen.
\subsection{MiniMax}
Die erste hier erläuterte Strategie ist der MiniMax Algorithmus. Dieser ist folgendermaßen definiert (siehe \cite{Russell.2016} S.164):
\\$\mathtt{MiniMax}(s,p) = \begin{cases} \mathtt{Utility}(s,p)$; wenn TerminalTest(s) $\\\mathtt{max}(\{\mathtt{MiniMax}(\mathtt{result}(s,a) \mid a \in \mathtt{actions}(s)\},(p +1)\%2 )\\ $; 
wenn Spieler am Zug$\\\mathtt{min}(\{\mathtt{MiniMax}(\mathtt{result}(s,a) \mid a \in \mathtt{actions}(s) \}, (p +1)\%2)\\ $; wenn Gegner am Zug$\end{cases}$
%\\$\mathtt{MiniMax}$ gibt in einem Endzustand einen Zahlenwert (-1,0,1) über das Spielende zurück.
%\\In allen anderen Zuständen gibt die Funktion eine Aktion zurück, welche den besten Spielzug darstellt.
%\\$MiniMax = \begin{cases} Q \times \mathtt{Player} \rightarrow \{-1,0,1\}$; wenn $\mathtt{Q} \in \mathtt{TerminalStates} \\ Q \times \mathtt{Player} \rightarrow \mathtt{actions}$ sonst $\end{cases}$
\\$\mathtt{getBestMove}: Q \times \{-1,0,1\} \rightarrow \mathtt{actions}$
\\$\mathtt{getBestMove}(s,\mathtt{score}, p) \\= \hfill \{ a \mid a \in \mathtt{actions} \land \mathtt{MiniMax}(\mathtt{result}(s,a), (p + 1) \% 2 ) == \mathtt{score}\}.\mathtt{any}()$
\\Die Funktion $\mathtt{getBestMove}$ ermittelt eine Menge aller Züge, welche den von $\mathtt{MiniMax}$ zurückgegeben Wert besitzen und wählt aus dieser Menge einen Zug aus.
\\Der Spieler sucht durch diese Funktion den bestmöglichen Zug aus den verfügbaren Zügen ($\mathtt{actions}$), der ihm einen für seine Züge einen Vorteil schafft, aber gleichzeitig nur \mxZitat{schlechte} Zugmöglichkeiten für den Gegner generiert. Der Gegner kann dadurch aus allen ehemals möglichen Zügen nicht den optimalen Zug spielen, da dieser in den aktuell enthaltenen Zügen nicht vorhanden ist. Er  wählt aus den verfügbaren $\mathtt{actions}$ nach den gleichen Vorgaben seinen besten Zug aus.
\vspace{0.5cm}\\ Die Strategie ist eine Tiefensuche und erkundet jeden Knoten zuerst bis zu den einzelnen Blättern bevor ein Nachbarknoten ausgewählt wird. Dies setzt das mindestens einmalige Durchlaufen des gesamten Suchbaumes voraus. Bei einem durchschnittlichen Verzweigungsfaktor von $f$ bei einer Tiefe von $d$ resultiert daraus eine Komplexität von $\mathcal{O}(d\textsuperscript{f})$. Bei einem einmaligen Erkunden der Knoten können die Werte aus den Blättern rekursiv von den Blättern zu den Knoten aktualisiert werden. Dadurch muss im nächsten Zug nur das Minimum aus $\mathtt{actions}$ ermittelt werden, da alle Kindknoten schon evaluiert wurden. Für übliche Spiele kann die MiniMax-Strategie allerdings nicht verwendet werden, da die Komplexität zu hoch für eine akzeptable Antwortzeit ist und der benötigte Speicherplatz für die berechneten Zustände sehr schnell wächst.

\subsection{\abab}
\label{ab-pruning}
Der MiniMax Algorithmus berechnet nach dem Prinzip \mxZitat{depth-first} stets den kompletten \gtree. 
\\Bei der Betrachtung des Entscheidungsverhaltens des Algorithmus fällt jedoch schnell auf, dass ein nicht unerheblicher Teil aller möglichen Züge durch einen menschlichen Spieler gar nicht erst in Betracht gezogen wird. Dies geschieht aufgrund der Tatsache, dass diese Züge in einem schlechteren Ergebnis resultieren würden als ein anderer, letztendlich ausgewählter, Zug.\newline
Dem \abab\ (\abp) Algorithmus liegt der Gedanke zugrunde, dass die Zustände, die in einem realen Spiel nie ausgewählt würden, auch nicht berechnet werden müssen. Damit steht die dafür regulär erforderliche Rechenzeit und der entsprechende Speicher dafür zur Verfügung andere, vielversprechendere, Zweige zu verfolgen.
\subsubsection{Demonstration an einem Beispiel}
\begin{figure}[ht!]
\caption[]{Beispielhafter \gtree}
\begin{adjustbox}{max width=\textwidth}
\Tree 
[.{A} 
	[.{B} 
		[.{E\\\color{grey}5} ].{E\\\color{grey}5} 
		[.{F\\\color{grey}13} ].{F\\\color{grey}13} 
		[.{G\\\color{grey}7} ].{G\\\color{grey}7} 
	].{B}
	[.{C}
		[.{H\\\color{grey}3} ].{H\\\color{grey}3}
		[.{I\\\color{grey}24} ].{I\\\color{grey}24}
		[.{J\\\color{grey}42} ].{J\\\color{grey}42} 
	].{C}
	[.{D} 
		[.{K\\\color{grey}42} ].{K\\\color{grey}42}
		[.{L\\\color{grey}6} ].{L\\\color{grey}6}
		[.{M\\\color{grey}1} ].{M\\\color{grey}1} 
	].{D} 
].{A}
\end{adjustbox}
\\\begin{adjustbox}{max width=\textwidth}
\Tree 
[.{A\\$\alpha = -\infty$ $\beta = +\infty$} 
	[.{B\\$\alpha = -\infty$ $\beta = 5$} 
		[.{E\\5} ].{E\\5} 
		[.{F\\\color{grey}13} ].{F\\\color{grey}13} 
		[.{G\\\color{grey}7} ].{G\\\color{grey}7} 
	].{B\\$\alpha = -\infty$ $\beta = 5$} 
	[.{C\\\color{grey}$\alpha = ?$ $\beta = ?$} 
		[.{H\\\color{grey}3} ].{H\\\color{grey}3}
		[.{I\\\color{grey}24} ].{I\\\color{grey}24}
		[.{J\\\color{grey}42} ].{J\\\color{grey}42} 
	].{C\\\color{grey}$\alpha = ?$ $\beta = ?$}
	[.{D\\\color{grey}$\alpha = ?$ $\beta = ?$} 
		[.{K\\\color{grey}42} ].{K\\\color{grey}42}
		[.{L\\\color{grey}6} ].{L\\\color{grey}6}
		[.{M\\\color{grey}1} ].{M\\\color{grey}1} 
	].{D\\\color{grey}$\alpha = ?$ $\beta = ?$} 
].{A\\$\alpha = -\infty$ $\beta = +\infty$}
\Tree 
[.{A\\$\alpha = -\infty$ $\beta = +\infty$} 
	[.{B\\$\alpha = -\infty$ $\beta = 5$} 
		[.{E\\5} ].{E\\5} 
		[.{F\\13} ].{F\\13} 
		[.{G\\\color{grey}7} ].{G\\\color{grey}7} 
	].{B\\$\alpha = -\infty$ $\beta = 5$} 
	[.{C\\\color{grey}$\alpha = ?$ $\beta = ?$} 
		[.{H\\\color{grey}3} ].{H\\\color{grey}3}
		[.{I\\\color{grey}24} ].{I\\\color{grey}24}
		[.{J\\\color{grey}42} ].{J\\\color{grey}42} 
	].{C\\\color{grey}$\alpha = ?$ $\beta = ?$}
	[.{D\\\color{grey}$\alpha = ?$ $\beta = ?$} 
		[.{K\\\color{grey}42} ].{K\\\color{grey}42}
		[.{L\\\color{grey}6} ].{L\\\color{grey}6}
		[.{M\\\color{grey}1} ].{M\\\color{grey}1} 
	].{D\\\color{grey}$\alpha = ?$ $\beta = ?$} 
].{A\\$\alpha = -\infty$ $\beta = +\infty$}
\end{adjustbox}
\\\begin{adjustbox}{max width=\textwidth}
\Tree 
[.{A\\$\alpha = 5$ $\beta = +\infty$} 
	[.{B\\$\alpha = 5$ $\beta = 5$} 
		[.{E\\5} ].{E\\5} 
		[.{F\\13} ].{F\\13} 
		[.{G\\7} ].{G\\7} 
	].{B\\$\alpha = 5$ $\beta = 5$} 
	[.{C\\\color{grey}$\alpha = ?$ $\beta = ?$} 
		[.{H\\\color{grey}3} ].{H\\\color{grey}3}
		[.{I\\\color{grey}24} ].{I\\\color{grey}24}
		[.{J\\\color{grey}42} ].{J\\\color{grey}42} 
	].{C\\\color{grey}$\alpha = ?$ $\beta = ?$}
	[.{D\\\color{grey}$\alpha = ?$ $\beta = ?$} 
		[.{K\\\color{grey}42} ].{K\\\color{grey}42}
		[.{L\\\color{grey}6} ].{L\\\color{grey}6}
		[.{M\\\color{grey}1} ].{M\\\color{grey}1} 
	].{D\\\color{grey}$\alpha = ?$ $\beta = ?$} 
].{A\\$\alpha = 5$ $\beta = +\infty$}
\Tree 
[.{A\\$\alpha = 5$ $\beta = +\infty$} 
	[.{B\\$\alpha = 5$ $\beta = 5$} 
		[.{E\\5} ].{E\\5} 
		[.{F\\13} ].{F\\13} 
		[.{G\\7} ].{G\\7} 
	].{B\\$\alpha = 5$ $\beta = 5$} 
	[.{C\\$\alpha = -\infty$ $\beta = 3$} 
		[.{H\\3} ].{H\\3}
		[.{I\\\color{grey}24} ].{I\\\color{grey}24}
		[.{J\\\color{grey}42} ].{J\\\color{grey}42} 
	].{C\\$\alpha = -\infty$ $\beta = 3$}
	[.{D\\\color{grey}$\alpha = ?$ $\beta = ?$} 
		[.{K\\\color{grey}42} ].{K\\\color{grey}42}
		[.{L\\\color{grey}6} ].{L\\\color{grey}6}
		[.{M\\\color{grey}1} ].{M\\\color{grey}1} 
	].{D\\\color{grey}$\alpha = ?$ $\beta = ?$} 
].{A\\$\alpha = 5$ $\beta = +\infty$}
\end{adjustbox}
\\\begin{adjustbox}{max width=\textwidth}
\Tree 
[.{A\\$\alpha = 5$ $\beta = 42$} 
	[.{B\\$\alpha = 5$ $\beta = 5$} 
		[.{E\\5} ].{E\\5} 
		[.{F\\13} ].{F\\13} 
		[.{G\\7} ].{G\\7} 
	].{B\\$\alpha = 5$ $\beta = 5$} 
	[.{C\\$\alpha = -\infty$ $\beta = 3$} 
		[.{H\\3} ].{H\\3}
		[.{I\\\color{grey}24} ].{I\\\color{grey}24}
		[.{J\\\color{grey}42} ].{J\\\color{grey}42} 
	].{C\\$\alpha = -\infty$ $\beta = 3$}
	[.{D\\$\alpha = -\infty$ $\beta = 42$} 
		[.{K\\42} ].{K\\42}
		[.{L\\\color{grey}6} ].{L\\\color{grey}6}
		[.{M\\\color{grey}1} ].{M\\\color{grey}1} 
	].{D\\$\alpha = -\infty$ $\beta = 42$}  
].{A\\$\alpha = 5$ $\beta = 42$}
\Tree 
[.{A\\$\alpha = 5$ $\beta = 5$} 
	[.{B\\$\alpha = 5$ $\beta = 5$} 
		[.{E\\5} ].{E\\5} 
		[.{F\\13} ].{F\\13} 
		[.{G\\7} ].{G\\7} 
	].{B\\$\alpha = 5$ $\beta = 5$} 
	[.{C\\$\alpha = -\infty$ $\beta = 3$} 
		[.{H\\3} ].{H\\3}
		[.{I\\\color{grey}24} ].{I\\\color{grey}24}
		[.{J\\\color{grey}42} ].{J\\\color{grey}42} 
	].{C\\$\alpha = -\infty$ $\beta = 3$}
	[.{D\\$\alpha = 1$ $\beta = 1$} 
		[.{K\\42} ].{K\\42}
		[.{L\\6} ].{L\\6}
		[.{M\\1} ].{M\\1} 
	].{D\\$\alpha = 1$ $\beta = 1$}  
].{A\\$\alpha = 5$ $\beta = 5$}
\end{adjustbox}
\end{figure}
Um den Algorithmus zu verdeutlichen betrachten wir das, an \cite{Russell.2016} angelehnte, folgende Beispiel. Das dargestellte Spiel besteht aus lediglich zwei Zügen, die abwechselnd durch die Spieler gewählt werden. An den Knoten der untersten Ebene des \gtree\ werden die Werte der Zustände gemäß der $\mathtt{utility}$ Funktion angegeben. Die Werte $\alpha$ und $\beta$ geben den schlecht möglichsten bzw. den bestmöglichen Spielausgang für einen Zweig, immer aus der Sicht des beginnenden Spielers, an. Die ausgegrauten Knoten wurden noch nicht betrachtet. Die Bäume werden von oben nach unten und links nach rechts durchlaufen.\\
Betrachten wir nun den linken Baum in der zweiten Zeile: 
\\Der Algorithmus beginnt damit alle möglichen Folgezustände bei der Wahl von B als Folgezustand zu evaluieren. Dabei wird zunächst der Knoten E betrachtet und damit der Wert 5 ermittelt. Dies ist der bisher beste Wert. Er wird als $\beta$ gespeichert. Ein Wert für $\alpha$ wurde bisher noch nicht ermittelt. \\
Im nachfolgenden \gtree\ wird der nächste Schritt verdeutlicht. Es wird der Knoten F betrachtet. Dieser hat einen Wert von 13. Am Zuge ist jedoch der zweite Spieler. Dieser wird, geht man davon aus, dass er ideal spielt, jedoch keinen Zug wählen, der ein besseres Ergebnis für den Gegner bringt als unbedingt nötig. Der bestmögliche Wert für den ersten Spieler bleibt damit 5.
\\Nach der Auswertung des Knotens G steht fest, dass es keinen besseren und keinen schlechteren Wert aus Sicht des ersten Spielers gibt. Daraufhin wird die 5 auch als schlechtester Wert in $\alpha$ gespeichert. Ausgehend von A  ist der schlechteste Wert damit 5 ggf. kann jedoch noch ein besseres Ergebnis herbeigeführt werden. $\alpha$ wird entsprechend gesetzt und $\beta$ verbleibt undefiniert.\\
Nun werden die Kindknoten von C betrachtet. Mit einem Wert von 3 wäre der Knoten H das bisher beste Ergebnis für die Wahl von C. Der Wert wird entsprechend gespeichert. Würde C gewählt gäbe man dem Gegenspieler die Chance ein im Vergleich zu der Wahl des Knotens B schlechteres Ergebnis herbeizuführen. Da Ziel des Spielers jedoch ist, die eigenen Punkte zu maximieren, gilt es diese Chance gar nicht erst zu gewähren. Entsprechend werden die Auswertung der weiteren Knoten abgebrochen.\\
Der Kindknoten K des Knotens D ist mit einem Wert von 42 vielversprechend und wird in $\beta$ gespeichert. Da dieser Wert größer ist als die gespeicherten 5 wird auch der entsprechende Wert von A aktualisiert. Der anschließend ausgewertete Knoten L ermöglicht nun ein schlechteres Ergebnis von 6 $\beta$, muss also aktualisiert werden. Der Knoten M liefert schließlich den schlechtesten Wert von 1. Da der Gegenspieler im Zweifel diesen Wert wählen würde, bleibt der bisher beste Wert das Ergebnis in E. In A wird der Spieler daher B auswählen.
\vspace{0.5cm}\\ 
Dieses einfache Beispiel zeigt bereits recht gut wie die Auswertung von weiteren Zweigen vermieden werden kann. In der Praktischen Anwendung befinden sich die wegfallenden Zustände häufig nicht nur in den Blättern des Baumes sondern auch auf höheren Ebenen. Der eingesparte Aufwand wird dadurch häufig noch größer.  

\subsubsection{Implementierung}
\label{ab-basics-impl}
Nachfolgend wird eine Pseudoimplementierung einer Invariante des \abab\ Algorithmus angegeben (siehe Listing \ref{lst:abprun}). Es handelt sich um eine angepasste Version von \cite{StroetmannAI19}.
\begin{lstlisting}[caption = {Pseudoimplementierung von \abab}, language = python, captionpos = t , numbers=left, label={lst:abprun}]
alphaBeta(State, player, alpha = -1, beta = 1) {
        if (finished(State)) {
            return utility(State, player)
        }
        val := alpha
        for (ns in nextStates(State, player)) {
            val = max({ val, -alphaBeta(ns, other(player), -beta, -alpha) })
            if (val >= beta) {
                return val
            }
            alpha = max({ val, alpha })
        }
        return val
}
\end{lstlisting}
Bei der angegebenen Implementierung handelt es sich um eine rekursive Umsetzung. Nachfolgend sei das Programm erleutert:
\begin{enumerate}
\item Im Basisfall wurde bereits ein Blatt des Spielbaumes erreicht. Damit ist das Spiel bereits beendet. In diesem Fall kann mit $utility$ der Wert des Zustands $State$ für den entsprechenden Spieler $player$ zurückgegeben werden.
\item In der Variable $val$ wird der maximale Wert aller von $State$ erreichbaren Zustände, sofern $player$ einen Zug ausführt, gespeichert.\\
Da der Algorithmus per Definition alle Wertigkeiten kleiner $alpha$ ausschließen soll, kann die Variable mit $alpha$ initialisiert werden.
\item Nun wird über alle Folgezustände $ns$ aus der Menge $nextStates(State, player)$ iteriert.
\item Nun wird rekursiv jeder Zustand $ns$ ausgewertet. An dieser Stelle ist jedoch der andere Spieler an der Reihe. Entsprechend erfolgt dies für den anderen Spieler. Da es sich um ein Nullsummenspiel handelt ist der Wert eines Zustandes aus Sicht des Gegners von $player$ genau der negative Wert der Wertigkeit für $player$. Aus diesem Grund müssen die Rollen von $alpha$ und $beta$ vertauscht und außerdem die Vorzeichen invertiert werden.
\item Da laut der Spezifikation des Algorithmus nur die Wertigkeit von Zuständen berechnet werden sollen, in denen diese kleiner oder gleich $beta$ ist, wird die Auswertung aller Folgezustände mit einem $val$ der größer oder gleich $beta$ ist abgebrochen. In diesem Fall wird $val$ zurückgegeben.
\item Sobald ein Folgezustand mit einem größeren Wert als $alpha$ gefunden wurde, kann $alpha$ auf den entsprechenden Wert erhöht werden. Sobald klar ist, dass der Wert $val$ erreicht werden kann, so sind Werte kleiner als $val$ nicht mehr relevant.  
 
\end{enumerate}

\subsubsection{Ordnung der Züge}
Wie in obigen Beispiel an den Zweigen unter dem Knoten C zu sehen war kann, je nach der Reihenfolge in der die Folgezüge untersucht werden, die Auswertung eines Folgezustandes früher oder später abgebrochen werden. Optimalerweise werden die besten Züge, also jene Züge die einen möglichst frühen Abbruch der Betrachtung eines Knotens herbeiführen zuerst betrachtet. Um dies abschätzen zu können bedient man sich in der Praxis einer Heuristik, die Aussagen über die Güte eines Zuges im Vergleich zu den übrigen Zügen zulässt. Anhand dieser Heuristik kann dann die Reihenfolge der Auswertung einzelner Folgezustände dynamisch angepasst werden. In Kapitel \ref{Heuristiken} werden unterschiedliche Heuristiken erklärt.

\subsection{Suboptimale Echtzeitentscheidungen}
Selbst die gezeigten Verbesserung des MiniMax-Algorithmus besitzt noch einen wesentlichen Nachteil. Da es sich um einen \mxZitat{depth-first} Algorithmus handelt muss jeder Pfad bis zu einem Endzustand betrachtet werden um eine Aussage über den Wert des Zuges treffen zu können. Dem steht jedoch die Tatsache entgegen, dass in der Praxis eine Entscheidung möglichst schnell, idealer Weise innerhalb weniger Sekunden, getroffen werden soll. Hinzu kommt die Tatsache, dass viele Spiele unter Verwendung von derzeit erhältlicher Hardware (noch) nicht lösbar sind.\\
Es gilt also eine Möglichkeit zu finden, die Auswertung des kompletten Baumes zu vermeiden.

\subsubsection{Heuristiken}
\label{Heuristiken}
Dieses Problem lösen sogenannte Heuristiken. Dabei handelt es sich um eine Funktion die versucht den Wert eines Spielzustandes anhand einzelner Eigenschaften des Zustandes anzunähern. Wie in Kapitel \ref{strat1} erläutert, hat ein Spieler der eine Ecke des Feldes besetzt in der Regel einen Vorteil. Ein solcher Zustand würde durch die Heuristik entsprechend besser bewertet werden.\\
Kommt eine Heuristik zur Anwendung, so ist die Genauigkeit, mit der diese den tatsächlichen Wert approximiert der wesentliche Aspekt der die Qualität des Spiel-Algorithmus ausmacht. Jedoch wird die Berechnung der Heuristik mit steigender Genauigkeit meist komplizierter und somit auch rechen- und damit zeitintensiver. Aus diesem Grund muss immer eine Abwägung aus Genauigkeit und Geschwindigkeit vorgenommen werden. 

\subsubsection{Abschnittskriterium der Suche}
\label{ab-depth-limited}
Gibt die Heuristik im Falle eines Endzustandes den Wert der Utility Funktion zurück, so kann die oben gezeigte Implementierung so angepasst werden, dass statt der Utility Funktion einfach die Heuristik ausgewertet wird. Dadurch muss nicht mehr der Vollständige Zweig durchsucht werden und das Abbrechen nach einer gewissen Suchtiefe wird möglich.
 
\subsubsection{Vorwärtsabschneiden}
Vorwärtsabschneiden (Forward Pruning) durchsucht nicht den kompletten \gtree , sondern durchsucht nur einen Teil. Eine Möglichkeit ist eine Strahlensuche, welche nur die \mxZitat{besten} Züge durchsucht (vgl. \cite{Russell.2016} S. 175). Die Züge mit einer geringen Erfolgswahrscheinlichkeit werden abgeschnitten und nicht bis zum Blattknoten evaluiert. Durch die Wahl des jeweiligen Zuges mit der höchsten Gewinnwahrscheinlichkeit können aber auch sehr gute bzw. schlechte Züge nicht berücksichtigt werden, wenn diese eine geringe Wahrscheinlichkeit besitzen. Durch das Abschneiden von Teilen des \gtree\ wird die Suchgeschwindigkeit deutlich erhöht. Der in dem \ot -Programm \mxZitat{Logistello} verwendete \mxZitat{Probcut} erzielt außerdem eine Gewinnwahrscheinlichkeit von 64\% gegenüber der ursprünglichen Version ohne Vorwärtsabschneiden (vgl. \cite{Russell.2016} S. 175).
\subsubsection{Suche gegen Nachschlagen}
\label{lookup}
Viele Spiele kann man in 3 Haupt-Spielabschnitte einteilen:
\begin{itemize}
\item Eröffnungsphase
\item Mittelspiel
\item Endphase
\end{itemize}
In der Eröffnungsphase und in der Endphase gibt es im Vergleich zum Mittelspiel wenige Zugmöglichkeiten. Dadurch sinkt der Verzweigungsfaktor und die generelle Anzahl der Folgezustände. In diesen Phasen können die optimalen Spielzüge einfacher berechnet werden. Eine weitere Möglichkeit besteht aus dem Nachschlagen des \states\ aus einer Lookup-Tabelle.
\\Dies ist sinnvoll, da gewöhnlicherweise sehr viel Literatur über die Spieleröffnung des jeweiligen Spiels existiert. % Auch über die Endzustände in der Schlussphase des Spiels findet sich Literatur. 
Das Mittelspiel jedoch hat zu viele Zugmöglichkeiten, um eine Tabelle der möglichen Spielzüge bis zum Spielende aufstellen zu können. Denn in diesem Spielabschnitt existieren üblicherweise 4 bis 12 Zugmöglichkeiten. Der in dem Kapitel \ref{othello-eroff} werden die bekanntesten Eröffnungsstrategien aufgelistet.
\\Viele Spielstrategien wie beispielsweise die MiniMax-Strategie setzen den kompletten oder wenigstens einen großen Teil des Spielbaums voraus. Dieser kann entweder berechnet werden oder aus einer Lookup-Tabelle gelesen werden. Je nach Verzweigungsfaktor der einzelnen Spielzüge kann diese allerdings sehr groß sein. Selbst im späten Spielverlauf gibt es verschiedene Spiele, welche einen großen Spielbaum besitzen.
\\Beispielsweise existieren für das Endspiel in Schach mit einem König, Läufer und Springer gegen einen König 3.494.568 mögliche Positionen (vgl. \cite{Russell.2016} S.176).
\\Dies sind zu viele Möglichkeiten um alle speichern zu können, da noch sehr viel mehr Endspiel-Kombinationen als diese existieren.
\\Anstatt die Spielzustände also zu speichern, können auch die verbleibenden Spielzustände berechnet werden. \ot\ besitzt gegenüber Schach den Vorteil, dass die Anzahl der Spielzüge auf 60 bzw. 64 Züge begrenzt sind. Dadurch kann in der Endphase des Spiel ggf. der komplette verbleibende \gtree\ berechnet werden, da die Anzahl der möglichen Zugmöglichkeiten eingeschränkt wird.
\\Bei der Berechnung der Spielzüge sind die Suchtiefe und der Verzweigungsfaktor entscheidend für die Berechnungsdauer. Aus diesem Grund können im Mittelspiel keine MiniMax-Algorithmen bis zu den Blattknoten des \gtrees\ ausgeführt werden, da die Menge des benötigten Speicherplatzes außerhalb jeglicher Grenzen eines Arbeits- oder Gamingscomputers liegen.

\section{\mc\ Algorithmus}
\label{mc_algo}
Im Gegensatz zu den bisher gezeigten Algorithmen verwendet der \mc\ Algorithmus einen stochastischen Ansatz um einen Zug auszuwählen. Im nachfolgenden Abschnitt wird die Funktionsweise des \mc\ Algorithmus erklärt. Daran angeschlossen folgen Möglichkeiten strategische Überlegungen zum Spiel \ot\ einzubringen. Dabei sind die nachfolgenden Ausführungen stark angelehnt an jene von \cite{nijssen_2007}.
\subsection{Algorithmus}
Als Ausgangspunkt legt der Monte Carlo Algorithmus die Menge der Züge zugrunde, die ein Spieler unter Wahrung der Spielregeln wählen kann. Diese Züge seien nachfolgend Zug-Kandidaten genannt. Enthält die Menge keine Züge, so bleibt dem Spieler nichts anderes übrig als auszusetzen. Enthält die Menge nur einen Zug, so muss der Spieler diesen ausführen. Per Definition ist dies dann der bestmögliche Zug. Enthält die Menge hingegen mindestens zwei mögliche Züge, so gilt es den besten unter ihnen auszuwählen. Um den besten Zug zu ermitteln, wird das Spiel mehrfach, die Anzahl sei $N_{P}$, bis zum Ende simuliert. Während der Simulation wird jeder mögliche Zug gleich häufig gewählt. Der Rest des simulierten Spiels wird dann zufällig zu Ende gespielt. Sobald alle Durchgänge erfolgt sind, wird das durchschnittliche Ergebnis für jeden möglichen Zug berechnet. Dabei gibt es zwei Möglichkeiten dieses Ergebnis zu berechnen:\\
Wahlweise kann die durchschnittliche Punktzahl eines Zuges oder die durchschnittliche Anzahl an gewonnenen Spielen herangezogen werden. Jener Zug, der nun das beste Ergebnis verspricht wird gespielt.
\subsection{Überlegungen zu \ot}
Bisher spielt der Algorithmus auf gut Glück ohne sich jeglicher Informationen des Spiels zu bedienen. In der Hoffnung das Spiel des Algorithmus zu verbessern, werden nun weitere Informationen zu \ot\ herangezogen. Hier sein zwei Möglichkeiten beschrieben um dies zu erreichen:
\paragraph{Vorverarbeitung}Wie in entsprechenden Kapitel gezeigt, gibt es strategisch gute und strategisch eher schlechte Züge. In seiner Reinform betrachtet der Monte Carlo Algorithmus jedoch beide Arten von Zügen gleich stark. Die Idee der Methode der Vorverarbeitung besteht darin, schlechte Züge in einem Vorverarbeitungsschritt auszuschließen um diese in den Simulationen gar nicht erst zu spielen. Um zu entscheiden, welche Züge ausgeschlossen werden, werden die einzelnen Spielzustände nach Ausführung des Zuges bewertet. Dazu werden die im entsprechenden Kapitel beschriebenen Kategorien von Spielsteinen herangezogen und mit einem entsprechenden Punktwert belegt. Für die Entscheidung an sich kann nun zwischen zwei Strategien gewählt werden:\\
Entweder kann eine feste Anzahl, diese sei $N_{S}$, an best bewertesten Züge ausgewählt werden oder alternativ eine variable Anzahl. Dies geschieht in dem der Durchschnitt aller Bewertungen bestimmt wird und nur jene Züge ausgewählt werden die eine gewisse Bewertung relativ zum Durchschnittswert haben. Dies wird in einer prozentualen Erfüllung des Durchschnittswertes, diese sei $p_{s}$, angegeben.
\paragraph{Pseudozufällige Zugauswahl}
In der Standardversion des Monte Carlo-Algorithmus werden die simulierten Spiele nach der Wahl des ersten Zuges zufällig zu Ende gespielt. Dem hier beschriebenen Ansatz liegt die Idee zu Grunde auch in dieser Phase der Simulation einige Züge anderen gegenüber zu bevorzugen. Dies geschieht nach dem gleichen Prinzip wie im Abschnitt zur Vorverarbeitung beschrieben. Da es jedoch sehr zeitaufwändig ist, die Bewertung jedes einzelnen Spielzustandes innerhalb der Simulationen vorzunehmen, erfolgt dies nur bis zu einer bestimmten Tiefe. Diese sei $N_{d}$.