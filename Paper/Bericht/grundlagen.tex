\chapter{Grundlagen}
\section{Spieltheorie}

\begin{Definition}[Game]
Ein \blue{Game} besteht aus einem Tupel der Form \\[0.2cm]
  \hspace*{1.3cm}
  $\mathcal{G} = \langle S\textsubscript{0},\mathtt{player}, \mathtt{actions}, \mathtt{result}, \mathtt{terminalTest}, \mathtt{utility} \rangle$
\\\so\ beschreibt den Startzustand des Spiels.
\\\textsc{player} ist auf der Menge der Spieler definiert und gibt den aktuellen Spieler zurück.
\\\textsc{actions} gibt die validen Folgezustände eines gegeben Zustands zurück.
\\\textsc{result} definiert das Resultat einer durchgeführten Aktion a und in einem Zustand s.
\\\textsc{terminalTest} prüft ob ein Zustand s ein Terminalzustand, also Endzustand, darstellt.
\\\textsc{utility} gibt einen Zahlenwert aus den Eingabenwerten s ( Terminalzustand) und p (Spieler) zurück. \\Positive Werte stellen einen Gewinn, negative Werte einen Verlust dar.
\info{quelle S.162}
\end{Definition}
Eine spezielle Art von Spielen sind \blue{Nullsummenspiele}.
\begin{Definition}[Nullsummenspiele]
In einem \blue{Nullsummenspiel} ist die Summe der utility Funktion eines Zustands über alle Spieler 0. Dies bedeutet, dass wenn ein Spieler gewinnt mindestens ein Gegenspieler verliert.
\end{Definition}
Durch den Startzustand \so\ und der Funktion \textsc{action} wird ein \blue{Spielbaum (Game tree)} aufgespannt.
\begin{Definition}[Spielbaum (Game Tree)]
Ein \blue{Spielbaum} besteht aus einer einzigen Wurzel, welche einen bestimmten Zustand (meistens \so) darstellt. Die Kindknoten der Wurzel stellen die durch \textsc{actions} erzeugten Zustände dar. Die Kanten zwischen der Wurzel und den Kindknoten stellen jeweils die durchgeführte Aktion dar, die ausgeführt wurde um vom State s zum Kindknoten zu gelangen.
\end{Definition}
\begin{Definition}[Suchbaum (Search Tree)]
Ein \blue{Suchbaum} ist ein Teil des Spielbaums.
\end{Definition}
\info{Überleitung einfügen}
\section{Spielstrategien}
Es gibt verschiedene Spielstrategien. Im Folgenden werden diese kurz erläutert und anschließend verglichen.
\subsection{MinMax}
Der erste hier erläuterte Strategie ist der MinMax Algorithmus. Dieser ist folgendermaßen definiert:

